\documentclass{article}
\usepackage{neurodata}

\title{What is Lifelong Learning?}
\author{Joshua T.~Vogelstein, Raman Arora, and Carey E.~Priebe}
\affil{Johns Hopkins University}
\date{December 2018}

\begin{document}

\maketitle

% \begin{abstract}
The goal of this document is to formally define lifelong learning from the perspective of statistical decision theory and learning theory.    
% \end{abstract}

\setcounter{tocdepth}{1}
\tableofcontents

\clearpage
\setcounter{section}{-1}
\section{Lifelong Learning Theory}
\subsection{Statistical Decision Theory}

\emph{Statistical decision theory} is a framework for formalizing decision theoretic tasks in the language of statistics.  An \emph{exploitation task} $\mathcal{T}$ is a sextuple defined by six elements.
\begin{enumerate}
  \item \textbf{Measurement Space:} $\mathcal{Z} \defn \{ \mathcal{Z}_k \forall k \in [K]\}$. We observe data in $\mathcal{Z}$, which is the composition of  $K$ modalities (e.g., images and text).  Samples in the measurement space are denoted  $z \in \mathcal{Z}$.  
  %
	\item \textbf{Density:} $P \in \mc{P}=\{P_{\theta}: \theta \in \Theta\}$.  A model $\mc{P}$ is a family of densities $P := P_{\theta}$, where the dimensionality of the parameter characterizing the distribution, $d=|\Theta|$, need not be finite.     We assume that random variables $Z \colon \Omega \to \mathcal{Z}$  are distributed according to some density $P$, with realizations $z \in \mathcal{Z}$.
    %
	\item \textbf{Action space:} $\mc{A}=\{a : a \in \mc{A}\}$.  The action space $\mc{A}$ is simply the set of potential actions.  Actions  include hypothesis testing, estimation, and control. 
	%
	\item \textbf{Decision function space:} $\Phi = \{\phi \colon \Xi^t \to \mc{A}, \forall t \}$. The decision function $\phi$ maps from a subset of measurements $\mathcal{Z} \subseteq \mathcal{Z}^t \subseteq \Xi^t$ to an action, and the decision function space $\Phi$ is the set of admissible decision functions.
	\item \textbf{Loss function:} $\ell \colon \mc{A} \times \mc{Z} \to \Real_+$.  The loss function $\ell \in \mc{L}$ scores each action for a given input.  Standard loss functions include the $0-1$ loss and squared error loss. Sometimes it will be convenient,  to allow $\ell$ to operate on $\Phi$, rather than $\mc{A}$. This is because each $\phi \in \Phi$ outputs an $a \in \mc{A}$, so $\ell(\phi,z)$ is simply interpretted as $\ell(a,z)$, where $a$ is the output of $\phi$.
    \item \textbf{Risk functional:} $\RR \colon \mc{L} \times \mc{P} \times \Phi \to \Real_+$.	The risk functional $\RR$ defines the criterion of optimality.  Classically, risk is defined as expected loss, $\RR_\ell(P,\phi) \defn \EE_P[\ell(Z,\phi(Z))]$. More recently, risk is defined as one minus the probability that loss is smaller than some constant $\eps>0$: $\RR(P,\phi)= 1-\PP_P[ \ell(Z,\phi(Z)) \leq \eps]$. 
\end{enumerate}

\noindent  A \emph{Bayes (or Optimal) decision rule} for a given exploitation task $\mathcal{T}=\{\mathcal{Z}, P, \mathcal{A}, \Phi, \ell, \RR\}$ is the decision rule that minimizes risk for that task: 
\begin{align}
    \phi_*^\mathcal{T} = \argmin_{\phi \in \Phi} \RR_\ell(P,\phi).
\end{align}
The resulting risk is called \emph{Bayes Risk} and is denoted $\RR_*^{\mc{T}}$.  A Bayes rule is not necessarily unique for a given task, and a given rule might be Bayes optimal for a number of different tasks.  That said, for any task, there exists a set of Bayes rules.  


\subsection{Learning Theory}

% TODO: raman - make this section formal/concise/correct/etc.

In learning theory we assume that we observe $T$ samples (or trials), $z_1,\ldots, z_T$, each of which is sampled identically and independently from some density $P$, that is $z_t \iid P$.  Denote the data corpus by $z^T =\{z_1,\ldots, z_t\} \in \mc{Z}^T$.  
We then devise  \textbf{learners}, $\mathcal{F} = \{f^t : \mathcal{Z}^t \to \Phi, \forall t\}$, that map $t$ measurements to an induced decision function, $\mh{\phi}_t$. 
A \emph{learning rule} is the sequence of learning functions, $\mv{f}= f_1,f_2,\ldots$.  A learning rule is \emph{consistent for model $\mc{P}$} if the risk of its estimated decision function converges to Bayes risk:
% \begin{align}
$\mv{f}_* \defn \{ \mv{f} : \RR_\ell(P,f_t(Z^t)) \to \RR_* \text{ as } t \conv \infty, \forall P \in \mc{P} \}$.
% \end{align}
A \emph{universally consistent learning rule} is a learning rule that is consistent for any model $\mc{P}$. 
%
The ``arbitrary slow convergence theorems''~\cite{Devroye1997-bd}  asserts that for a given learning rule $\mv{f}$, and number of trials $t$, there exists a distribution $P$, such that the loss of $\mh{\phi}_t$ is larger than $\eps > 0$ with  probability larger than $1-\delta$, for $\delta>0$.   Based on this, learning theory, rather than dealing with asymptotic results, is concerned with finite sample properties for a given space of decision functions, $\Phi$, or a given model $\mathcal{P}$. 
The key metric in  learning theory is therefore \emph{regret} (or \emph{excess loss}), that is, the cumulative loss relative to the Bayes (or Oracle) learner:
\begin{align}
    R(\mh{\phi}) = \sum_{t=1}^T \ell(\mh{\phi}_t, z_t) - \inf_{f \in \mathcal{F}} \sum_{t=1}^T \ell ( f(Z^T), z_t),
\end{align}
where we have dropped the subscript on $\phi_T$ for brevity. 
Because the samples  are random, so too will be regret.  Therefore, rather than minimizing empirical regret, the goal is to minimize risk, which is some function of regret, such as expected regret. 
% \begin{align}
%     \mathcal{R}_\ell(P^T,\phi^T) = \sum_{t=1}^T \EE_{P_t}[R_T(\phi_t)].
% \end{align}
% 
Alternately, one may desire to maximize the probability that  regret is smaller than a constant:
\begin{align} \label{eq:goal}
% \left\lvert \sum_{t=1}^T \ell(\phi_t, z_t) - \inf_{f \in \mathcal{F}} \sum_{t=1}^T \ell ( \phi(f), z_t) \right\rvert
    \PP_{ z_1,\ldots z_T \iid P} \left[R(\mh{\phi})  \leq \eps \right] > 1 - \delta.
\end{align}
A decision function space $\Phi$ is said to be \emph{``probably almost correct'' (PAC) learnable} when there exists a class of learners $\mc{F}$ that achieves~\eqref{eq:goal} with a  number of trials that is polynomial in (i) the dimensionality of $z$, $p=|\mc{Z}|$, (ii) the ``computational cost'' $c$ of representing any $\phi \in \Phi$, (iii) $\eps$, and  (iv) $\delta$.  A learning rule is said to be \emph{efficient} if it achieves~\eqref{eq:goal} with only polynomial space and/or time with respect to $\kappa= \{p,c,\eps,\delta\}$.  

\subsection{Online Learning and Transfer Learning}

% TODO: raman - make this section formal/concise/correct/etc.


\emph{Online learning}~\cite{Mohri2018-tf} and \emph{transfer learning}~\cite{Pan2010-bv} can both be thought of as generalizations of learning.  In online learning, learning happens sequentially, as data arrive.  The key difference, however, is that we no longer assume that each trial is sampled identically and independently from some distribution, that is, we no longer assume $Z_t \iid P$.  Rather, we assume $Z_t \sim P_t$, and $P_t$ is allowed to changed, potentially smoothly, or even adversarialy.  

In transfer learning, there are explicitly multiple tasks. 
Recall that a given task $\mathcal{T}$ is uniquely specified by a sextuple, $\{\mc{Z}, P, \mc{A}, \Phi, \ell, \mc{R}\}$.  In transfer learing, we have a \emph{multitask} $\mathcal{T}'$, which  consists of a set of tasks.  Without loss of generality, we consider here the simplest construction: the multitask is a tuple of two tasks, $\mc{T}'=\{\mc{T}_0, \mc{T}_1\}$,  a source task $\mathcal{T}_0$, and a target task $\mathcal{T}_1$, and at least one of six elements has changed between the two tasks.  Either the source or target task could be a multitask; the most general form is that both the source and target tasks are multitasks. 


Define $\mc{T}_j=\{\mc{Z}_j, P_j, \mc{A}_j, \Phi^j, \ell_j, \RR_j\}$, for $j \in \{0,1\}$.  Assume  $W \iid P_0$ where $w \in \mathcal{Z}_0$ and 
$Z \iid P_1$ where $z \in \mathcal{Z}_1$.  
And let $T_0$ and $T_1$ denote the sample size of the source and target data corpori, respectively. 
There also made be some side information, for example, constraints on how the task has changed, or advise from $N$ experts. Let side information be abstract, $s \in \mc{S}$. \textbf{Lifelong learners}, $\mc{F}^{L2}=\{ f^t: \mc{Z}_0^{T_0} \times \mc{Z}_1^t \times {S}^{T_0+t} \to \Phi_1, \forall t\}$ map $T_0$ measurements from the source task and $t$ measurements from the target task and side information for all trials,  to an induced decision function, $\mh{\phi}_t^1$ (below we clarify why these learning rules are called lifelong learners). 

Regret for transfer learning corresponds to the excess loss on $\mc{T}_1$ relative to a learning rule with access to $Z^0$:
\begin{align}
    R^{TL}(\mh{\phi}) = \sum_{t=1}^{T_1} \ell_1(\mh{\phi}^1_t, z_t)    - \inf_{f \in \mc{F}} \sum_{t=1}^{T_1} \ell_1(f(W^{T_0}, Z^{T_1}, S^{T_0+T_1}), z_t).
\end{align}
The goal in transfer learning  is to choose learning rules that satisfy:
\begin{align} \label{eq:transfer_goal}
    \PP\left[R(^{TL}\mh{\phi})  \leq \eps \right] > 1 - \delta,
\end{align}
for large classes of multitasks, 
where the probability above is taken with respect to $W_s \sim P_0^s$ and $Z_t \sim P_1^t$ for all $s \leq T_0$ and $t \leq T_1$.  Note that the above construction, in fact, also holds for online learning.  Let $\mc{T}_0$ be the multitask characterizing the trials up to and including $T_0$, and $\mc{T}_1$ characterize the \emph{next} trial.  Then the above perfectly describes online learning.  The relationship between lifelong learning, and both online learning and transfer learning will be discussed in more detail below. 

A \emph{transfer learning machine} (TLM) is any learning rule that is designed to solve a transfer learning multitask. A \emph{consistent TLM} is a TLM that asymptotically achieves Bayes risk on the target task $\mc{T}_1$.  
A \emph{universally consistent TLM} is an TLM that is consistent for any multitask $\mc{T}'$.  
An \emph{efficient TLM}  is a consistent TLM that achieves Bayes risk with polynomial space and time with respect to the number of trials. 
A multitask is \emph{PAC transfer learnable} when there exists a class of learners $\mc{F}^{L2}$ that achieves~\eqref{eq:transfer_goal} with a number of samples polynomial in $\kappa_0$ and $\kappa_1$, where $\kappa_j=p_j,c_j,\eps_j,\delta_j$.  Similarly, a multitask is \emph{efficiently lifelong learnable} when there exists a class of learners $\mc{F}^{L2}$ that achieves~\eqref{eq:transfer_goal} with only polynomial space and/or time with respect to $\{\kappa_0, \kappa_1\}$.
{Transfer learning theory is concerned with identifying the conditions under which transfer learning multitasks are (not) PAC-learnable, and are (not) efficient.}



\subsection{Lifelong Learning}

% TODO: raman - make this section formal/concise/correct/etc.


Lifelong learning generalizes and extends online learning and transfer learning.  \textbf{The key difference between lifelong learning, and both online learning and transfer learning, is forgetting}.  Specifically, in lifelong learning, we are not only concerned with loss on the target task, but rather, we are also concerned with loss on the source task.  Specifically, in lifelong learning, regret is defined as the excess loss on both the source and the target task, where the learning rule can use all the data from the source task for the target task:
% \begin{multiline} \label{eq:l2m_goal}
%     R^{L2}(\mh{\phi}) = 
%     \sum_{t=1}^{T_1} \ell_1(\mh{\phi}^1_t, z_t)   
%     % 
%     - \inf_{f \in \mc{F}} \sum_{t=1}^{T_1} \ell_1(f(W^{T_0}, Z^{T_1}, S^{T_0+T_1}), z_t) \\
%     % 
%     + \sum_{t=1}^{T_0} \ell_0(\mh{\phi}^0_t, w_t)  
%     % 
%     - \inf_{f \in \mc{F}} \sum_{t=1}^{T_0+T_1} \ell_0(f(W^{T_0}, Z^{T_1}, S^{T_0+T_1}), w_t)  
%     % \nonumber \\
%     % &= R^{TL}(\mh{\phi}) 
%     % + \sum_{t=1}^{T_0+T_1} \ell_0(\mh{\phi}^0_t, w_t) 
%     % % 
%     % - \inf_{f \in \mc{F}} \sum_{t=1}^{T_0+T_1} \ell_0(f(W^{T_0},Z^{T_1} S^{T_0}), w_t) 
% \end{multiline}

\begin{multline*}
    R^{L2}(\mh{\phi}) =     \underbrace{\sum_{t=1}^{T_1} \ell_1(\mh{\phi}^1_t, z_t)       - \inf_{f \in \mc{F}} \sum_{t=1}^{T_1} \ell_1(f(W^{T_0}, Z^{T_1}, S^{T_0+T_1}), z_t)}_{\text{transfer learning regret}} + \\
     \underbrace{\sum_{t=1}^{T_0+T_1} \ell_0(\mh{\phi}^0_t, w_t)   - \inf_{f \in \mc{F}} \sum_{t=1}^{T_0+T_1} \ell_0(f(W^{T_0}, Z^{T_1}, S^{T_0+T_1}), w_t)  }_{\text{forgetting regret}},
\end{multline*}
which is simply the sum of transfer learning regret with \emph{forgetting regret}.  Note that forgetting regret includes the excess loss on the source task over all trials, including those trials corresponding to the target task.  This is because in lifelong learning, it is important that performance on the source target does not decrease as the learner is updating for the target task, and the loss on the source target could even continue to improve with additional target task trials (called \emph{backwards transfer}).
The goal in lifelong learning is similar to the goal in learning theory and transfer learning, to find lifelong learners that satisfy achieve low regret with high probability for a large class of multitasks:
\begin{align} \label{eq:l2m_goal}
    \PP\left[R(^{L2}\mh{\phi})  \leq \eps \right] > 1 - \delta.
\end{align}
A \emph{lifelong learning machine} (L2M) is any learning rule that is designed to solve a lifelong learning multitask. A \emph{consistent L2M} is an L2M that asymptotically achieves Bayes risk on the both tasks $\mc{T}_0$ and $\mc{T}_1$.  
A \emph{universally consistent L2M} is an L2M that is consistent for any multitask $\mc{T}'$.  
An \emph{efficient L2M}  is a consistent L2M that achieves Bayes risk with polynomial space and time with respect to the number of trials. 
A multitask is \emph{PAC lifelong learnable} when there exists a class of learners $\mc{F}^{L2}$ that achieves~\eqref{eq:l2m_goal} with a number of samples polynomial in $\kappa_0$ and $\kappa_1$.  Similarly, a multitask is \emph{efficiently lifelong learnable} when there exists a class of learners $\mc{F}^{L2}$ that achieves~\eqref{eq:l2m_goal} with only polynomial space and/or time with respect to $\kappa_0$, and $ \kappa_1$.
{Lifelong learning theory is concerned with identifying the conditions under which lifelong learning multitasks are (not) PAC-learnable, and are (not) efficient.}




There are two flavors of lifelong learning: unsupervised and supervised. 
In \textbf{unsupervised lifelong learning} (UL2), the lifelong learning rule knows the task \emph{may} change, but does not know when or how. The unsupervised case is only appropriate when the density $P$ has changed, because if any of the other elements change, the learning rule \emph{must} know if it could even try to perform well. 
The only difference between online learning and UL2 is that UL2 is also concerned with loss on the source task. UL2 is also sometimes called \emph{continual learning}~\cite{noauthor_undated-cl}.


In \textbf{supervised lifelong learning} (SL2), the learning rule knows that the task has changed, but still does not know  the Bayes rule for either task.  The supervised case is more general, as any of the six task elements could change. Note that ``supervised'' here refers to whether the fact that the task has switched is known to the learning rule, not whether the task itself is supervised.  The only difference between supervised lifelong learning and transfer learning is that SL2 is also concerned with loss of the source task. 



It would seem that the UL2 is more difficult than the SL2, but the latter has its own unique challenges as well. We provide further details on UL2  in Section~\ref{sec:continual}, and SL2 in Section~\ref{sec:transfer}.  
Section~\ref{sec:perception} describes how a comprehensive L2M must incorporate a mechanism for detecting when and how the task changes.  Section~\ref{sec:struct} describes learning the task structure, which can improve L2M peformance, and Section~\ref{sec:safe} describes how one can ensure that a given L2M is safe.  



 

% In other words, there is no single decision rule that is universally consistent.  However, it may still be the case that  lifelong learning is concerned with cases that are not PAC-learnable.  






% \clearpage 
\section{Unsupervised Lifelong Learning}
\label{sec:continual}

There are several increasingly complex flavors unsupervised lifelong learning (UL2) that we describe below.  We start by describing a scenario that is closely related to lifelong learning, called sequential learning, in which the distribution is fixed, but the decision rule is learned sequentially, rather than in ``batch mode'' with access to all the data at once. 



\subsection{Sequential Learning}
\label{sec:constant}

% TODO: raman - make this section formal/concise/correct/etc.


In sequential learning, the learner only observes a single measurement at a time, and must update the decision function based on that knowledge.  In other words, for each $t = 1,\ldots, T$, the learner is $f^t \from \mathcal{Z}^t \to \Phi$.  
%  
The goal of sequential learning is to find a sequence of learners, $f_1, \ldots, f_T$, that minimizes the expected regret for a given unknown density $P$.  The reason the above statement is non-trivial is because the implications are that the learner must be sufficiently good for all $t<T$ such that the resulting loss is not much more than it would have been if the learner had access to all $T$ observations even for the first $t<T$ observations.
% 
The approach to this setting, ignoring computational complexity, is trivial.  At each $t$, simply learn a new decision rule on all of the data. One can do much better than that computationally.  Specifically, one can sequentially update the learned decision rule~\cite{Xiao2010-gv}. Strategies for sequential learning are a natural starting place for certain lifelong learning tasks.



\subsection{Discrete but Dynamic $P_t$}
\label{sec:jumps}

The simplest unsupervised lifelong learning setting is the case when $P_t$ can change $L < \infty$ times.  Dealing with this situation requires estimating change points, and the way one does so depends on whether $L$ is provided.  We assume that $L$ is not provided, as that is a more difficult setting.  Because  the only difference between $\mc{T}_j$ and $\mc{T}_j'$ is $P_j \neq P_{j'}$, the goal can be dramatically simplified:
\begin{align} \label{eq:online}        
    \PP_{ z_t \sim P_t} \left[R(\mh{\phi})  \leq \eps \right] > 1 - \delta, \text{ where } P_t \in \mc{P}_T \subsetneq \mc{P}, \quad |\mc{P}_T| = L,
\end{align}
where here and in the sequal we drop the superscript $^{L2}$ for brevity. In this setting, to obtain theoretical results, we require some assumptions about the $P_t$'s, so that we can evaluate the performance of a lifelong learner with regard to an oracle that knows $P_t$ for all $t$.  For example, we can assume that these densities are sampled from some ``base distribution'', $P_t \sim P$. When $L$ is (approximately) known, the L2M can use that information to determine how large an estimated change between $P_j$ and $P_{j'}$ justifies asserting that the task has changed.  Without this information, there is an additional \emph{model selection} problem, akin to estimating the number of Gaussians in a Gaussian mixture model~\cite{Fraley2002-rm}.  Solving these model selection problems is notoriously difficult, and requires some additional \emph{side information} to obtain theoretical guarantees, such as the desired trade-off between model fit and model complexity.  Regardless, the key to solving these change point detection problems is having a mechanism to quantify the difference between $P_j$ and $P_{j'}$. 




\subsection{Continuously Dynamic $P_t$}
\label{sec:smooth}

A  more complicated unsupervised lifelong learning setting is the case when $P_t$ is continuously changing.  This setting is often called ``online learning'' in the learning theory community, when one is only concerned about loss on the subsequent tasks. Here, the learning rule must either explicitly or implicitly address the fact that $P_t$ is  changing over time.  The current dominant approach in machine learning to address this situation is to assume access to advise from $N$ experts~\cite{Mohri2018-tf}. Note that the goal of continuously dynamic UL2 is the same as that for discrete by dynamic UL2.  The only difference is is the constraints on $P_t$:
\begin{align} \label{eq:online}        
    \PP_{ z_t \sim P_t} \left[R(\mh{\phi})  \leq \eps \right] > 1 - \delta, \text{ where } P_t \in \mc{P}_T \subsetneq \mc{P}, 
\end{align}
and there may be additional constraints on how $P_t$ is allowed to change.

In this setting obtaining theoretical results also requires some additional assumptions about the sequence $\{P_t\}$.
For example, one could assume that $P_t$ follows  first order Markov chain dynamics, meaning that $P_t$ is independent of $P_{t-j}$  for all $j=2,3,...$ when given $P_{t-1}$.  More generally, one can simply assume that the dynamics on $P_t$ are relatively smooth, that is, $\norm{P_t - P_{t-1}} < c$. It is also important here to be able to determine the magnitude of the change from $P_t$ to $P_{t+1}$. 



\subsection{Changing Model $\mc{P}$}
\label{sec:model}

In the most complex  UL2 problem  the model changes, or is simply too large for a single lifelong learner to have any performance guarantees. A model is ``too large'' whenever there does not exist a universally consistent learning rule, which implies that the task is not PAC-learnable. We can model these tasks as a multitask of learnable tasks, in other words, the multitask contains a large number of tasks, each of which is PAC learnable.  The L2M must therefore somehow filter the multitask down to a learnable task, possibly using expert advise, and then learn just the relevant remaining information.  We provide two examples, one is a standard in machine learning, the so-called ``universal concept class''~\cite{Mohri2018-tf}, and the other comes from a recent publication on vertex nomination~\cite{Lyzinski2017-jh}.  

The universal concept class can be described as follows.  Consider the set $\mc{X}=\{0,1\}^p$ of all Boolean vectors with $p$ components, and let $\mc{C}$ be the concept class formed by all subsets of $\mc{X}$.  To guarantee that a Bayes decision rule is within the decision rule space we require that $|\Phi| \geq |\mc{C}| = 2^{2^p}$, because there are $2^p$ possible bitstrings in $\mc{X}$, and therefore the power-set of $\mc{X}$ includes $2^{2^p}$ elements.  Based on this, it is trivial to show a sample complexity bound:
\begin{align}
    p \geq \frac{1}{\epsilon} \big(  (\log 2) 2^p + \log \frac{1}{\delta} \big),
\end{align}
which indicates that the required number of samples is exponential in $p$, meaning that it is not PAC-learnable.  Equivalently, without making additional assumptions, such as $p < \infty$, there does not exist a universally consistent decision rule. 

% TODO: vince - make the below paragraph precise/formal/correct/etc.


Lyzinski et al.~\cite{Lyzinski2017-jh} recently showed the existence of another machine learning task for which there does not exist a universally consistent decision rule: vertex nomination (VN).  In VN, we sample two correlated graphs, for which a subset of the vertices are shared.  Given a set of vertices of interest in one graph, we desire to rank the vertices in the other graph such that the highest ranked vertices are the vertices that correspond to the vertices of interest in the first graph.  They demonstrated that no decision rule is universally consistent for this task, without imposing additional prior knowledge or constraints.   The intuition for why this is the case is that as the sample size $t$ increases, so does the model complexity, and therefore the complexity of consistent decision rules, at an exponential rate.


One can, of course, construct other example tasks or multitasks that lack universal consistent decision rules.  For example, if the model (or decision function space) increases faster than sample size, in general, there is no universally consistent decision rules.  Thus, solving these problems requires incorporating additional knowledge.  As suggested above, one approach is to assume the existence of $N$ experts, who offer advise after each sample.  Another is to try to determine what is the space of decision functions for which a given learning rule is consistent (its consistency class). 
Thus, in general, to solve any of the above multitasks, the goal is to: (1) find the appropriate consistency class (possibly using expert advise), and then (2) minimize risk within that class.  How to find appropriate consistency classes  in practice remains an open question.


% \clearpage
\section{Supervised Lifelong Learning}
\label{sec:transfer}

As described above,  a supervised lifelong learning multitask, $\mc{T}$, is composed of two tasks, $\mc{T}_0$, and $\mc{T}_1$.  The fact that it is supervised indicates that for any given sample, the learner is aware of which task the sample corresponds to.  In other words, at each trial, the learner receives the tuple $(z_t, j_t) \in \mc{Z} \times [64]$, where $j_t$ is the task at trial $t$, and there are $64$ possible different flavors of SL2, each corresponding to whether or not one of the six elements of the target tast differs from the source task.    
Thus, the key that makes a given task lifelong learning, rather than classical learning, is that it is possible that the Bayes decision rule associated with the source task is different from the target task.  The difference could arise due to differences in: the measurement spaces, the distributions, the action spaces, the loss function, or the risk functional, because the Bayes decision rule is a function of each of these things.  Below we address each of these possibilities.

% TODO: jovo - add content to all the below subsections of SL2

\subsection{Changing Measurement Space}

Formally, assume that each $\mt{\mc{Z}}_0=\Xi \times \mc{Z}_0'$ and $\mt{\mc{Z}}_1 = \Xi \times \mc{Z}_1'$.  However, rather than directly observing data sampled from $\mt{\mc{Z}}_j$, we observe data after some unknown (potentially nonlinear) transformation has been applied to each, $\rho_j \from \mt{\mc{Z}}_0 \to \mc{Z}_j$. The goal, therefore, is to find $\Xi$, the joint representational space for both task.  There are two distinct scenarios, either of which is possible when $\mc{Z}_0 \neq \mc{Z}_1$. 

In the simplest scernario,  $\rho_j$ is the identity function, and $\mc{Z}_0 = \Real^p$ and $\mc{Z}_1 = \Real^{p'}$.  In this case, the L2M can find the intersection of the dimensions of the two tasks, and simply operate on those dimensions.  This option could be effective when the cardinality of the intersection is large, relative to $p$ or $p'$.  


Another strategy that is applicable much more generally,  is to apply a \emph{manifold matching} technique, that is, to find an embedding, $\eta \from \mc{Z}_0 \times \mc{Z}_1 \to \Xi$, such that given a new sample $z_t$, regardless of whether there is information in $mc{Z}_0$ and/or $\mc{Z}_1$, it can be embed into $\Xi$~\cite{Shen2017-ou}.  This is perhaps the most prominent strategy in recent applications, especially in deep learning, with impressive results.  

A common strategy for manifold matching proceeds as follows.  First, $n_0$ samples in $\mc{Z}_0$ and $n_1$ samples in $\mc{Z}_0$, define an interpoint comparison function for each space, $\rho_0$ and $\rho_1$.  Second, compute the $n_0 \times n_0$ and $n_1 \times n_1$ interpoint comparison matrices.  Third, search for an embedding of these matrices that jointly optimizes some desirable cost function.  Canonical components analysis is the simplest and most common variant of this approach~\cite{Hardoon2004-cg}, though more recent proposals including distance correlation~\cite{Cowley2017-ll} and Hilbert-Schmidt Information Criterion~\cite{Chang2013-iz} based approaches have emerged as promisting alternatives. 



\subsection{Changing Density}

When $P_0 \neq P_1$, an L2M  has several options, and which to take depends on the nature of the difference.  Let $\alpha=\norm{ P_0 - P_1}$ be a suitable norm over the true distances between the densities.  In that case, consider four scenarios:
\begin{enumerate}
    \item The difference between the distributions is quite small, that is, $\alpha \ll \min (\norm{P_0},\norm{P_1} )$.  In this case, the L2M should operate as if the two densities are the same, as the bias incurred by virtue of assuming they are the same would be less costly than the variance incurred by assuming they are different, and having to estimate both densities.
    \item The difference between the distributions is quite large, that is, $\alpha \gg \max (\norm{P_0},\norm{P_1} )$. In this case, the L2M should operate as if the two densities are unrelated, as the bias incurred by virtue of not parameter sharing whatsoever would be smaller than the variance incurred by assuming some joint distribution should be estimated.
    \item The difference between the distributions is moderate, $\min (\norm{P_0},\norm{P_1} ) < \alpha < \max (\norm{P_0},\norm{P_1} )$.  In this case, parameter sharing is warranted. 
    \item $P_0 \in \mc{P}$ but $P_1 \notin \mc{P}$.  This is the scneario closely associated with UL2 challenge of changing model $\mc{P}$ discussed in Section~\ref{sec:model}, and is most difficult. 
\end{enumerate}

The key to success in the changing density scenario is similar to the UL2 challenges.  Specifically, we require a mechanism to quantify the degree of difference between the two distributions.  This question can formally be posed as a hypothesis test:
\begin{align*}
    H_0 &: P_0 = P_1 \\
    H_A &: P_0 \neq P_1.
\end{align*}

There are many previously proposed non-parametric, universally consistent,  two-sample tests that could be applied here.  Importantly, the test statistic for these tests quantifies the degree of difference between the two distributions.  A further requirement therefore is that the test statistic is monotonic with regard to the difference in the distributions.  Finally, to be generally applicable, the test must be applicable to data in arbitrary metric spaces.  Several distance~\cite{Rizzo2010-vr} and kernel~\cite{Gretton2012-ue} methods could therefore satisfy these properties. 

Note, however, depending on the task, there may be lower dimensional spaces that contain all the \emph{task relevant information}, but discard other information.  This is the principle behind the information bottleneck principle~\cite{Tishby1999-tj}.  Thus, rather than quantifying the difference between $P_0$ and $P_1$, we quantify the difference between $I(Z^0,X)$ and $I(Z^1,X)$, where $X$ corresponds to the task relevant latent variable, and $I(\cdot)$ is Shannon information. 



\subsection{Changing Action Space, Decision Function Space, Loss Function, or Risk Functional}

If ``the Boss'' decides, for some reason, that the action space, decision function spae, loss function, or risk functional changes, then the L2M has a ``top down'' lifelong learning problem, rather than a ``bottom up'' change as discussed above.  This case is substantially under developed and studied.  In a certain sense, these are all easier, and more common.  For example, deep learning systems have often been ``pre-trained'' in an unsupervised fashion, prior to training them for the specific supervised task of interest. Moreover, for classification systems that provide a scaled output, such as a posterior probability of being in one class or another, it is relatively straightforward to convert those probabilities into scalar predictions (for a regression problem).  As another example, often the decision function space changes in deployment, if the resources become cut-back, requiring a more computationally efficient system.  While each of these cases can already be somewhat addressed,  addressing them in new multitasks will require further specialization and evelopment. 

Irrespective of which top-down change is imposed, much like in the other SL2 challenges, a mechanism for determining which change has occured, and adjusting the learning is required.  The next section describes this process in more detail. 


% Here we consider $\mc{A}_0 \neq \mc{A}_1$, for example, changing from classification to regression, or object detection to object localization.  

% \subsection{Changing Decision Function Space}

% more to come...

% \subsection{Changing Loss Function}

% more to come...

% \subsection{Changing Risk Functional}

% more to come...

% \subsection{Changing Multiple Elements}

% more to come...
% We now consider the case when there are multiple discrete tasks.  Formally, we say that there are two datasets, a source dataset $Z^0$ and a target dataset $Z^1$. The source dataset contains $n_0 = | Z^0|$ samples, and the target dataset contains $n_1 = |Z^1|$ samples. In general, these $n_0+n_1$ samples could come from a large number of tasks, but for simplicity, and without loss of generality, we will assume each is associated with a single task.  Each dataset is associated with a different task.  As described above, a task is a sextuple, each of the six different elements of the tasks could differ from one another. Thus, there are therefore $2^6=64$ different flavors of supervised lifelong learning (SL2). The multitask is supervised if and only if the learning rule knows that it faces a new task.



% The goal of SL2 it to improve the \emph{rate} of learning on the target task, given the source task. More formally, we assume:
% $Z \iid F_0$ where $z \in \mathcal{Z}_0$ and 
% $W \iid F_1$ where $w \in \mathcal{Z}_1$.  
% And let $T_0$ and $T_1$ denote the sample size of the source and target data corpori, respectively.  
% The goal is to choose a sequence of learners, $f$, such that the decision rules that they induce result in lower loss than would could achieve without using the source data:
% \begin{align}
%     R^S(\phi) = \sum_{t=1}^{T_0} \ell(\mh{\phi}_t, z_t)    - \inf_{f \in \mc{F}} \sum_{t=1}^{T_0} \ell(f(Z^{T_0},W^{T_1}), z_t).
% \end{align}
% The above supervised lifelong learning regret is a random variable because it is a function of the data, and therefore, we desire to maximize the probability that $R^S$ is smaller than some constant, $\eps>0$. 




\section{Goal Driven Perception}
\label{sec:perception}


% \subsection{The Agnostic Setting with Regard to Dynamics of $P_t$}
% \label{sec:combo}


In each of the above settings, we assumed that our L2M had  knowledge with regard to which of the  settings it was currently facing, either unsupervised or supervised, and then which of the sub-flavors it faced (for example, continuously dynamic online learning, or supervised learning where the measurement space changed).  The most general setting, however, is that the learner is not told which of the sub-flavors it is in.  In each case, learning requires a meta-learner, to determine the setting, and tell the learner what setting it believes it is in. 

For example, in UL2,  if $P_t$ is known to change discretely, then a L2M needs a mechanism by which it can estimate whether  new samples, $z_{t+1}, z_{t+2},\ldots$ are sampled from some $P_{t'} \neq P_t$, and if so, how different are $P_{t'}$ and $P_t$, which can be measured by some metric, $\norm{P_t - P_{t'}}$.  If the model is changing, the L2M needs a mechanism to estimate which consistency class to operate in. Alternately, in SL2, the L2M needs a mechanism to query the larger system.  For example, it could query ``the Boss'' to ascertain whether the action space, decision funcion space, loss function, or risk functional have changed.  Or it could query the measurement devices to see if they have run calibration experiments, that do not provide samples, but do provide ``side information'' that the L2M could use to infer whether the density of the measurements have changed, or whether the measurement space has changed (for example, a new sensor is installed).  

To formalize the above, we say that the metalearner has access to $N$ additional sources of side information.  Its job is to output which of the settings, $j_t \in \mc{J}$ the L2M should be operating in. It could operate in a probabilistic manner, in which it would output a probability distribution of the system being in any of the potential settings.  In SL2, $|\mc{J}|=64$, whereas in UL2, $|\mc{J}|=4$.  In certain settings, additional information is useful for the L2M system. For example, the effect size,  $\norm{P_t - P_{t'}}$, is useful for certain supervised and unsupervised settings. 



\section{Latent Task Structure Learning}
\label{sec:struct}

% TODO: avanti - make this section precise/formal/correct/etc., a la LSM paper

In all of the above settings, there are multiple tasks that characterize the multitask.  It can be useful to understand the latent structure organizing the different tasks.  For example, assume we have $\mc{T}_0, \mc{T}_1, \ldots, \mc{T}_J$, And the difference between $\mc{T}_0$ and $\mc{T}_j$ is merely a translation, meaning that $\phi_0^* = \phi_j^* + c_j$, where $c \in \Phi$. If we are able to estimate this multitask structure, then we can use this information to improve our regret for subsequent tasks, assuming they follow the same structure.  

For concretely, let's associate with each task a unique $\phi^*_j$.  Further, let's assume that the $\phi$'s are sampled from some distribution $F_\phi$, and that $\Phi \subseteq \RR^p$.  $\Phi$ might be a manifold, or a union of manifolds. Thus, given $Z^0, Z^1, \ldots, Z^J$, estimate the latent structure and $F_\phi$ on that latent structure.   



\section{Safe Learning}
\label{sec:safe}

% TODO: jovo - error correcting, calibration, etc.


In lifelong learning, everything can change.  A consequence of this is that the L2M can potentially behave quite erratically.  To ensure that the L2M does not exhibit such behaviors, certain ``safety checks'' are required.  These safety checks counter-act potential adversarial attacks. Adversarial attacks are particularly problematic in L2M's, because they can impact not just the action resulting from a single sample, but they can also change the system to lead to undesirable actions for future samples as well.    

There are two inter-related aspects to these safety checks.  First, the L2M must provide not just point estimates, but must also quantify its uncertainty for a given sample.  For example, these uncertainty quantifications could come in terms of posterior probabilities for classification problems, or confidence intervals, for estimation problems.  These metrics of uncertainty provide the user with information about whether the current sample is ``near'' the other previously observed samples.  If not, it might be adversarial.  Second, the L2M must quantify the degree of ``outlyingness'' of a given sample.  

Thus, for any new sample, the L2M should: (1) provide a an estimated best action and a confidence-score associated with each action, (2) provde an estimate of the outlyingness of the sample, (3) decide whether or not to update the system on the basis of the new sample, or withhold it until further information is acquired.  

The first of the above tasks, providing a confidence-score associated with each action, is built-in to any statistical and/or Bayesian L2M.  That said, there remains an issue called ``calibration''.  Specifically, many machine learning models provide such an output, but are not appropriately calibrated, and are therefore overly confident about certain inputs. To ensure that one obtains accurate estimates of uncertainty, rather than merely prediction, requires additional work.  This can be formalized by asserting that our goal, rather than finding the maxmimum likelihood estimator, or the maximum a posteriori estimator, to estimating \emph{quantiles}.  

To be more concrete and formal, consider a two-class classification problem.  In this case, the action space is binary, $\mc{A}=\{0,1\}$, corresponding to the two classes.  However, to ensure a properly calibrated posterior estimate, instead, the action space is the unit inteval, $\mc{A}=(0,1)$, which corresponds to the estimated posterior probability of being in class $1$. The loss functional must be suitably modified as well.  

The second of the above tasks, quantifying outlyingness, is closely related.  Continuing with the two-class example above, if the new sample has a low posterior probability of being in both classes, it is an outlier (where ``low'' is a threshold somebody must set).  However, typically in classification problems, the probability of being in each class is normalized, so that the total probability of being in any of the classes sums to one.  Thus instead, an L2M should also allow the possibility of a sample coming from a different, and unknown class, with its own distinct distribution.  This kind of problem is called ``supervised clustering'', because there are some samples with class labels, but also many others without, and the total number of classes is unknown.  More generally, we desire that the L2M quantifies the probability of a sample coming the the previously observed distribution.  

The third of the above tasks, deciding whether to incorporate the new sample, is a higher level decision.  Specifically, given an estimated outlyingness, and confidence associated with that outlyingness, ``the boss'' can decide on a function that determines whether to use and/or act on the new sample or not.  This is a question of risk mitigation, and therefore, a subjective choice, given the suitable knowledge.



\bibliographystyle{naturemag}
\bibliography{lifelonglearningbib}

\clearpage
\appendix

% TODO: jovo - do this appendix
more to come...

\section{LifeLong Learning Forests}

Here, we demonstrate how lifelong learning forests (L2F) can be developed into the first and only universal lifelong learning machines, that is, lifelong learning machines that achieve optimal performance over any distributional assumptions.

In the subsequent chapters,  for each of the five components of lifelong learning, we provide:
\begin{enumerate}
    \item our proposal for how lifelong learning forests can achieve efficiency, and in which contexts,
    \item a set of baseline algorithms, and
    \item a set of challenge problems
\end{enumerate}

\subsection{Continual Learning}

Our Continual Lifelong Learning Forest will be based on the following.  Recent work demonstrated that a certain class of decision trees is minimax optimal, meaning that the worst risk it achieves over all distributions $P \in \mathcal{P}$ is the smallest possible over all decision functions $\phi \in \Phi$~\cite{Scott2006-oo}.  This is achieved by writing down the solution to finding the best decision tree as a penalized optimization problem, where the penalty corresponds to the tree complexity (not just the tree depth)~\cite{Scott2005-ho}. In other work, a dual averaging method for regularized online learning was demonstrated to be effective~\cite{Xiao2010-gv}, which can be applied to learning the trees. Therefore, we propose to combine these two approaches to build collections optimal trees online, which we will combine into a single forest. 

We propose a few baseline comparisons, each of which provide upper bounds, although of different kinds.  First, we compare to the ``batch mode'' decision forest, where at each time $t$, we simply train an entire new forest.  In theory, our online forest will perform as well.  Some empirical evidence in other domains suggests that online performance can even improve batch performance, possibly because the online methods are less likely to get caught in local minima.  Second, we compare to the ``plug-in'' estimator.  Specifically, since we know $P$ on simultated data, we can use the data to directly estimate $P$, and then plug the answer in.  This is not using random forest at all, and will achieve better performance for certain $t$'s, depending on the dimensionality.  Third, we compare to the actual Bayes estimator, using the true $P$.  No learner can possibly ever do better than this, so this provides a strict upper bound on accuracy. 


\subsubsection*{Discretely Changing $P_t$}

To proceed, the learning rule must therefore devise a mechanism by which it can estimate whether a change point has occurred. Once it has determined that a change has occurred, then it can start building a new decision rule, and continue checking whether a change has occurred again.  If it detects another change, it then compare to all the previous ``epochs'', if the change actually corresponds to a previous setting, then it continues building the decision rule from that setting.  Otherwise, it again generates a new decision rule. In this section, we do not concern ourselves with the relationship between the different ``epochs'', but see Section~\ref{sec:transfer} for details about how to address this scenario.   

Given the above, all that remains in the discrete dynamic $P_t$ is to detect change points. We propose two different methods for doing so; technically, any method could be applied here. First, we could multiscale graph correlation (MGC)~\cite{Shen2016-fo, Shen2017-ub}.  In year 1 of Phase 1, we proved a number of properties of MGC, including that it is a universally consistent test, and that specifically it is universally consistent for testing for the equality of two distributions~\cite{Shen2018-st}.  However, to have a fully ``decision forest'' approach, we could instead do the following. 

It is well established in the statistics literature that mutual information (MI) can be a universally consistent test statistic for Independence testing.  Moreover, in year 1 of Phase 1, we proved that any independence test can also be a $K$-sample test, meaning that MI can also be a universal $K$-sample test. However, estimating MI in high-dimensional settings, such as those that motivate L2M, is notoriously difficult.  We propose to use decision forests to estimate MI.  More specifically,  MI is the difference of entropy and conditional entropy.  And since only conditional entropy changes, conditional entropy can be the test statistic.  Prior work proved that decision forests can result in consistent estimates for quantile regression~\cite{Meinshausen2006-kw}.  A generalization of this approach demonstrates that decision forests can also provide consistent estimates of conditional entropy, which is a functional of the set of quantiles.  In preliminary work (with an undergraduate), we have demonstrated that with a few tricks (using ``honest sampling'' and an estimation procedure that is robust to finite sampling variability), we can achieve consistent estimates of mutual information in high-dimensions with relatively low sample sizes (see Figure~\ref{fig:cond_entropy}). Thus, our L2F will leverage these results to perform change point detection. Upon proving the universal consistency of L2F to estimating mutual information, proving the universality of the whole approach (using decision forests to estimate change points, and using online learning forests within each epoch), will be straightforward. 

\begin{figure}
    \centering
    \includegraphics{cond_entropy.png}
    \caption{Empirical demonstration that a variant of decision forests provides a universally consistent estimate of conditional entropy, which can be used as a test statistic for change points.}
    \label{fig:cond_entropy}
\end{figure}

An important aspect of using L2F to estimate conditional entropy and mutual information, is that these statistics can also be used to quantify the ``effect size'', that is, the magnitude of the change.  In fact, many existing definitions of effect size are special cases of MI under certain restrictive parametric settings.  The magnitude of the effect size that we estimate will be important in other sections.


As a baseline, we will compare this approach to the Oracle approach, which knows when there is a change point.  The metrics would be both concerning the timing of change point detection (how long does it take L2F to detect the change point), and how does the delay impact regret. We will also compare to the parametric approach, which knows the change point time, and the model family, but not the precise distribution, so it must estimate the distribution after each change point. Finally, we hide the change point time from the above parametric approach. Doing so provides a bound on how well one could estimate change points, as compared to our approach. 

\subsubsection*{Continuously Changing $P_t$}

To do so using L2F, we adapt our continual learning L2F described above.  Specifically,~\citet{Cavallanti2007-sv} demonstrated how to track the best hyperplane with a simple budget perceptron.  We will incorporate this technique into each node of our L2F, to enable tracking changes. We will then prove that regardless of the dynamics, the regret of this L2F on any smoothly changing $P_t$, including dynamics produced by reactive environments, will be small.  

The baseline to evaluate this approach would again be the Oracle approach, which knows everything, and the parametric approach, which knows the model family, but does not know the specific parameters. 

\subsubsection*{Agnostic to Dynamics L2F}


Specifically, L2F in this context would work as follows.  For each new data point, $z_t$, L2F would continually update.  Moreover, L2F would also be checking whether the new sample, and/or the most recent $\tau$ samples are different from the previous epoch's samples using the approach described in Section~\ref{sec:jumps}.  The estimated effect size will be used to weight the amount that the L2F updates the decision boundary under the assumption that things are changing (using Section~\ref{sec:smooth}), versus under the assumption that things are not (using Section~\ref{sec:constant}).  

The baselines here will again be the Oracle, as well as the parametric method with varying degrees of task knowledge. 


\subsection{Supervised Lifelong Learning}

A recent publication proposed two mechanisms for using random forests for transfer learning~\cite{Segev2016-pu}, which is quite related to what we originally wrote in our L2F proposal. The two ideas essentially boil down to either (1) updating the structure of given trees, and (2) updating the parameters of the trees (e.g., the features and magnitudes).  The empirical results are impressive, though the theoretical results are limited.  Specifically, whether L2F's constructed in either of these ways achieve universal transfer learning remains an open question.  

Moreover, just because two different tasks are different, does \emph{not} mean that one \emph{should} update the induced decision rules.  Rather, the difference between the two tasks, and the relative amount of new data, dictates whether and how much one should transfer, versus either assuming there has been no difference (which incurs some bias but reduced variance), or ignoring the old data (which incurs no bias but increased variance).  Thus, strong theoretical claims about whether and how much one should transfer remain wide open questions, specifically in the L2F setting.  By virtue of the approach we proposed above to estimate the effect size (in Section~\ref{sec:jumps}), we will be able to ascertain how much mixing to perform, and prove the conditions under which an appropriate degree of mixing is warranted. 

The baselines, therefore, for this task 



\end{document}
